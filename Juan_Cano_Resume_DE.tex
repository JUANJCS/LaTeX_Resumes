% CV — JUAN CANO-SELLES 
% Zielposition: Projektmanager Digital Solutions & Process Automation (Sanitas)
% Sprache: Deutsch (Business Fluent) | Final Optimized Version

\documentclass[a4paper,10pt]{article} 
\usepackage[a4paper,margin=0.4in]{geometry} 
\usepackage{enumitem} 
\usepackage[hidelinks]{hyperref} 
\usepackage[ngerman]{babel} 
\usepackage{iftex} 

% ---------- Fonts ---------- 
\ifPDFTeX 
  \usepackage[T1]{fontenc} 
  \usepackage[utf8]{inputenc} 
  \usepackage[scaled]{helvet} 
  \renewcommand{\familydefault}{\sfdefault} 
\else 
  \usepackage{fontspec} 
  \setmainfont{Helvetica} 
\fi 

% Layout Spacing
\setlist[itemize]{leftmargin=0.15in, itemsep=0pt, topsep=1pt, parsep=0pt, partopsep=0pt} 
\linespread{1.0} 
\setlength{\parindent}{0pt} 

\newcommand{\Section}[1]{% 
  \vspace{7pt}% 
  {\large\bfseries\uppercase{#1}}% 
  \vspace{1pt}\par\noindent\hrule height 0.6pt% 
  \vspace{3pt}% 
} 

\pagestyle{empty} 

\begin{document} 

% ============================================================ 
% HEADER (Clean & Professional)
% ============================================================ 
\begin{center} 
    {\Huge\bfseries JUAN CANO-SELLES} \\ \vspace{6pt}
    
    % Administrative Infos kompakt in einer Zeile
    \textbf{Spanischer Staatsbürger (EU) | Wohnhaft in Deutschland | Umzugsbereit} \\ \vspace{2pt}
    
    % Kontaktinfos (etwas kleiner, um den Fokus auf die Kompetenzen zu lenken)
    \small +49 163 150 9581 | \href{mailto:juan.cano.selles@gmail.com}{juan.cano.selles@gmail.com} | \href{https://www.linkedin.com/in/jcanoselles}{linkedin.com/in/jcanoselles} \\ \vspace{2pt}
    
    % Die harten Fakten für den Job
    \textbf{Deutsch: C1 (Verhandlungssicher) | Englisch: C2 | Sofort verfügbar}
\end{center}

% ============================================================ 
% EXECUTIVE PROFILE
% ============================================================ 
\Section{Führungsprofil} 
\textbf{Technical Program Manager | Digital Solutions \& Process Optimization} \\ 
Senior Professional mit 10 Jahren Erfahrung (3 Jahre Dev / 7 Jahre Agile Lead) in der Bereitstellung digitaler Lösungen in hochregulierten Umfeldern. Experte für die Synchronisation von Unternehmensstrategie und technischer Exekution.
\begin{itemize} 
    \item \textbf{Agile Governance:} Steuerung von PI-Plannings und Release-Zyklen für mehrere IT-Squads zur Sicherstellung der Roadmap-Ziele.
    \item \textbf{Prozessautomatisierung:} Etablierung einer Cloud-Test-Engine: Automatisierung der Dokumentation und Supplier-Integration für eine 360-Grad-Echtzeitsicht auf die Release-Reife.
    \item \textbf{Schnittstelle Business-IT:} Übersetzung komplexer regulatorischer Anforderungen in technische Backlogs mittels dualer Kompetenz in Strategie und Architektur.
\end{itemize} 

% ============================================================ 
% EXPERIENCE 
% ============================================================ 
\Section{Berufserfahrung} 

\textbf{CARIAD SE} $|$ \textbf{Technical Program Manager / Senior PO} \hfill \textit{Wolfsburg, DE $|$ Nov 2020 -- Aug 2025} 
\begin{itemize} 
\item Koordination von 5 Squads (Cloud Portfolio Activation Cluster) in PI-Plannings und Release-Zyklen zur Sicherstellung der technischen Lieferziele.
\item Definition und Liefersteuerung einer cloud-basierten Testautomatisierungsplattform (Gauge/Jira X-Ray) zur direkten Ausführung von Testspezifikationen.
\item Durchsetzung von Supplier-Akzeptanztests in Deployment-Pipelines und vereinheitlichtem Reporting für eine automatisierte 360-Grad-Qualitätssicht.
\item Leitung der Implementierung der 'Functions on Demand'-Plattform: Rollout digitaler Pay-per-Use-Features für >4 Mio. vernetzte Fahrzeuge weltweit.
\item Leitung eines FinOps-Programms: >30\% AWS-Kostenreduktion (YoY) durch Ressourcen-Audits und Ausrichtung der Systemanforderungen auf den Business-Bedarf.
\item Stakeholder-Management zur Definition der Business-Architektur unter Einhaltung von GDPR- und Datensouveränitäts-Vorgaben.
\end{itemize} 

\vspace{6pt} 
\textbf{SEAT S.A.} $|$ \textbf{Technical Project Lead (Critical Systems Integration)} \hfill \textit{Barcelona, ES $|$ Mär 2019 -- Nov 2020} 
\begin{itemize} 
\item Technische Leitung der Integration des Connectivity-Steuergeräts, inkl. Release-Planung und Defect-Management mit Tier-1-Zulieferern.
\item Sicherstellung der Homologation sicherheitskritischer Funktionen (EU eCall) durch Koordination von Validierungskampagnen (HiL und Fahrversuch).
\item Cross-funktionale Integration der Connectivity-Units: Sicherstellung der Dokumentationsreife (Packaging, Verkabelung, SW-Varianten) und technisches Enablement der Schnittstellenbereiche.
\end{itemize} 

\vspace{6pt} 
\textbf{Akkodis} $|$ \textbf{Systems Integration Engineer} \hfill \textit{Barcelona, ES $|$ Okt 2015 -- Mär 2019} 
\begin{itemize} 
    \item Implementierung der Applikationslogik für bezahlnahe Systeme unter Einhaltung strenger Energie- und Datenintegritätsvorgaben.
    \item Entwicklung von Hardware-Emulatoren (Python/Qt) zur Beschleunigung der Protokollvalidierung für ressourcenbeschränkte Hardware.
    \item Tätigkeit als externer technischer Projektleiter für SEAT S.A. zur Steuerung von Integrationsliefergegenständen vor dem internen Stellenwechsel.
\end{itemize} 

\Section{Kernkompetenzen \& Tools} 
\begin{itemize} 
\item \textbf{Methoden:} Agil (SAFe/Scrum), PI-Planning, Backlog Prioritization, Requirements Engineering.
\item \textbf{Sprachen:} Spanisch (Muttersprache), Englisch (C2), Deutsch (C1/Verhandlungssicher), Französisch (A2--B1).
\item \textbf{Technische Domäne:} Workflow \& Process Automation (Jira X-Ray/CI/CD), Cloud Infrastructure (AWS), SDLC.
\item \textbf{Zertifizierungen:} PMP (2025), SAFe Product Owner/Product Manager (2021).
\end{itemize}

\Section{Ausbildung} 
\textbf{M.Sc. Telecommunications Engineering} \hfill Miguel Hernández Universität, Spanien | 2015 

\end{document}