% CV — JUAN J. CANO-SELLES 
% Technical Product Manager | Cloud Governance

\documentclass[a4paper,10pt]{article} 
\usepackage[a4paper,margin=0.4in]{geometry} 
\usepackage{enumitem} 
\usepackage[hidelinks]{hyperref} 
\usepackage{bookmark} 
\usepackage{iftex} 
\usepackage{microtype} 

% ---------- Fonts ---------- 
\ifPDFTeX 
  \usepackage[T1]{fontenc} 
  \usepackage[utf8]{inputenc} 
  \usepackage[scaled]{helvet} 
  \renewcommand{\familydefault}{\sfdefault} 
\else 
  \usepackage{fontspec} 
  \setmainfont{Helvetica} 
\fi 

% Executive Layout Spacing 
\setlist[itemize]{leftmargin=0.15in, itemsep=1pt, topsep=2pt, parsep=1pt, partopsep=0pt, before=\raggedright} 
\linespread{1.05} 
\setlength{\parindent}{0pt} 

\newcommand{\Section}[1]{% 
  \vspace{8pt}% 
  {\large\bfseries\uppercase{#1}}% 
  \vspace{1pt}\par\noindent\hrule height 0.6pt% 
  \vspace{4pt}% 
} 

\pagestyle{empty} 

\begin{document} 

% ============================================================ 
% HEADER 
% ============================================================ 
\begin{center} 
    {\Huge\bfseries JUAN J. CANO-SELLES} \\ \vspace{5pt}
    
    \textbf{Spanischer Staatsangehöriger (EU/EFTA) | Wohnsitzwechsel in die Schweiz kurzfristig möglich} \\ \vspace{3pt}
    
    +49 163 150 9581 | \href{mailto:juan.cano.selles@gmail.com}{juan.cano.selles@gmail.com} | \href{https://www.linkedin.com/in/jcanoselles}{linkedin.com/in/jcanoselles} \\ \vspace{3pt}
    
    \textbf{Verfügbarkeit: Sofort}
\end{center}

% ============================================================ 
% SUMMARY 
% ============================================================ 
\Section{Executive Profile} 
\textbf{Technical Product Manager | Cloud Infrastructure \& Governance} \\ 
10 Jahre Erfahrung im Systems Engineering, davon 6 Jahre in technischer Führung und 5 Jahre spezialisiert auf Cloud Governance und Skalierung von Hochverfügbarkeits-Plattformen.

\begin{itemize} 
    \item \textbf{Cloud Unit Economics:} Experte für die Übersetzung technischer Ressourcenverbräuche in betriebswirtschaftliche Metriken. Erfolgreiche Durchsetzung von FinOps-Strategien zur signifikanten Reduktion von Cloud-Unit-Costs durch Architekturoptimierung.
    \item \textbf{Interdisziplinäre Steuerung:} Erfahren in der Orchestrierung komplexer Schnittstellen zwischen Infrastruktur-Architektur, Engineering und Software-Entwicklung. Fokus auf effiziente Delivery-Pipelines ohne administrativen Overhead.
    \item \textbf{Technisches Fundament:} Engineering-Background in C/C++ und Linux-Systemintegration sichert fachliche Akzeptanz bei Senior Engineers und Kernel-Entwicklern für eine präzise Anforderungsanalyse.
\end{itemize}

% ============================================================ 
% EXPERIENCE 
% ============================================================ 
\Section{Berufserfahrung} 

\textbf{CARIAD SE} $|$ \textbf{Technical Program Manager (Senior Product Owner)} \hfill \textit{Wolfsburg, DE $|$ Nov 2020 -- Aug 2025} 
\begin{itemize} 
\item \textbf{Programmleitung:} Verantwortung für die End-to-End Delivery Roadmap und SLA-Einhaltung (99,9 \% Verfügbarkeit) eines Agile Release Train (5 DevOps-Teams) für eine Flotte von 4 Mio.+ Fahrzeugen.
\item \textbf{Cloud-Architektur:} Steuerung der Migration zu einer Event-driven Architecture (Kafka, SNS/SQS); Auflösung monolithischer Abhängigkeiten zur Ermöglichung asynchroner Skalierung.
\item \textbf{FinOps \& Kosteneffizienz:} Realisierung einer 30 \% Kostensenkung (AWS) durch architektonisches Redesign von Datenbank-Zugriffsmustern und Optimierung von Logging-Intervallen.
\item \textbf{Produkt-Launch:} Technische Leitung für die Einführung von Functions on Demand (FoD) und B2B-Flottendiensten; Integration komplexer digitaler Features in die Konzern-Infrastruktur.
\item \textbf{Governance \& Compliance:} Sicherstellung globaler regulatorischer Anforderungen (u. a. China RTM, EU eCall) und Umsetzung von Data-Residency-Vorgaben durch geografische Datentrennung.
\end{itemize} 

\vspace{6pt} 
\textbf{SEAT S.A.} $|$ \textbf{Technical Project Lead (Connectivity Systems)} \hfill \textit{Barcelona, ES $|$ März 2019 -- Nov 2020} 
\begin{itemize} 
\item \textbf{Release-Management:} Synchronisation von Hardware- und Software-Zyklen für Konnektivitäts-Steuergeräte (SOP-Sicherstellung für drei Fahrzeuglinien).
\item \textbf{Lieferantensteuerung:} Technisches Management von Tier-1-Lieferanten (LG); Priorisierung von Hardware-Revisionen für die Firmware-Validierung.
\item \textbf{Homologation:} Sicherstellung der Konformität für sicherheitskritische EU-eCall-Standards; Überwachung von Crashtests und Validierung der Datensicherheit.
\item \textbf{Troubleshooting:} Lösung von Integrations-Blockern im Technical Development Centre durch Entwicklung eigener Diagnose-Skripte zur Unterstützung der Validierungsteams.
\end{itemize} 

\vspace{6pt} 
\textbf{Akkodis} $|$ \textbf{Systems Integration Engineer} \hfill \textit{Barcelona, ES $|$ Okt 2015 -- März 2019} 
\begin{itemize} 
    \item \textbf{Embedded Systems:} Entwicklung von Applikationslogik und Firmware für IoT-Systeme unter strikten Energie- und Echtzeitanforderungen.
    \item \textbf{Tooling:} Programmierung kundenspezifischer Linux-Treiber und Hardware-Emulatoren (Python/Qt) zur Automatisierung von Validierungs-Workflows.
    \item \textbf{Consulting (SEAT S.A.):} Leitung der Software-Validierung und Hardware-in-the-Loop (HiL) Tests für Serien-Konnektivitätseinheiten.
\end{itemize} 

\Section{Sprachen \& Zertifizierungen} 
\begin{itemize} 
\item \textbf{Sprachen:} Spanisch (Muttersprache), Englisch (C2), Deutsch (B2 - Professionelle Arbeitskenntnisse).
\item \textbf{Zertifizierungen:} PMP (Project Management Professional, 2025), SAFe PO/PM (2021).
\end{itemize}

\Section{Ausbildung} 
\textbf{Master of Science in Telecommunications Engineering} \hfill Miguel Hernández University, Spanien | 2015

\end{document}